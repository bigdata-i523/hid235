\documentclass[sigconf]{acmart}

\usepackage{hyperref}

\usepackage{endfloat}
\renewcommand{\efloatseparator}{\mbox{}} % no new page between figures

\usepackage{booktabs} % For formal tables

\settopmatter{printacmref=false} % Removes citation information below abstract
\renewcommand\footnotetextcopyrightpermission[1]{} % removes footnote with conference information in first column
\pagestyle{plain} % removes running headers
\usepackage{indentfirst}
 
\begin{document}
\title{Big data: An Opportunity for Historians}


\author{Yujie Wu}
\affiliation{%
  \institution{Indiana University Bloomington}
  \city{Bloomington} 
  \state{Indiana} 
  \postcode{47401}
}
\email{yujiwu@iu.edu}



% The default list of authors is too long for headers}
\renewcommand{\shortauthors}{B. Trovato et al.}


\begin{abstract}
\large  In human history, catastrophe, wars, big event led to a incredible loss of life. Historians collected the data of life but ignored to do the statistical analysis due to their non-statistical background. In this new age, Big data provides an insight for historians to learn humanity based on the data from the history. This paper involves the brief introduction of historical events, big data analysis based on the historical information, and the results from the big data learning.  
\end{abstract}

\keywords{\large i523, HID235, history, Titanic, ID3}

\maketitle

\section{Introduction}
\large With the rapid development of information technology, historians now enter the age of big data. Big data refers to a tremendous 
amount of information that produced by human and nonhuman activities in the past. The scale of big data is large enough that it is 
impossible for individuals to collect and preprocess the data. Hence, history which is derived from human engagement with the past must
have some affinity with big data and the computer technology it represents\cite{C1}. Such instinctive property of big data provides an
absolutely new perspective for historians to study and re-evaluate the history. 

\par 
In this age of explosion of information, zillions of pieces information is stored on the Internet. The volume, velocity, variety, value, veracity of data are the treasure for historian to mine. But in most of time, data is neither straightforward substance nor transparent material for historians to squeeze the interesting information since they are not well-organized into a meaningful format that let the algorithm analyze them for answering the questions that historians are interested in\cite{C1}. Processing meaningless data into a coherent argument is not an easy job. Furthermore, using proper infrastructure and algorithm is difficult as well. As a historian, big data is an opportunity but also a big obstacle for the future researches.  

\section{Preprocess data}
 
\par
Preprocessing data is a big challenge for historians. Here is an example to illustrate a proper approach to preprocess history-relative data. The data to be introduced in this paper is from the world-famous tragedy -- Titanic. Titanic was one of three ``Olympic Class'' liners which were an incredible feat of engineering and ambition in their age. Titanic was the largest, fastest, and most luxurious liner. Its maiden voyage was from Southampton to New York with a lot of people on board including millionaires, movie stars, teachers and labors who were looking for a better life in United States. However, it struck an iceberg and sank in Atlantic Ocean five days after the beginning of its journey. The collision tore a series of holes along side of the hull. The sea water came into Titanic and less than three hours later, Titanic sank down about 2 miles to the bottom of the Atlantic ocean. Overall 1502 out of 2224 on broad passengers and crew lost their lives in this shocking tragedy\cite{C2}. 

\par
This tragedy attracted the attention of international community. People were wondering the reasons that how such technique marvel encountered this tragedy. One of the well-known reasons that the sinking of Titanic led to such loss of life was that there were not enough lifeboats for the passengers and crew\cite{C3}. Another reason was that on the night of Sunday April 14 1912, the Atlantic ocean was flat calm, the sky clear and moonless, and the temperature was freezing-cold\cite{C2}. The weather condition was very difficult for captain and other crew to detect an iceberg. Therefore, such weather condition explained the reason why the alarm of iceberg in front was made only 40 seconds before Titanic crashed the iceberg. It was impossible for such big mechanical monster to provide a stop response. Unfortunately, Titanic accelerated towards to iceberg directly and tore a series of large holes along side of the hull.  

\par
The information from the passengers and crew on board was collected later on for historians to study one more interesting field that what sorts of people were possibly to survive. 

\par
The description of the data used to study for historians in this example is as follows. The data set has 12 attributes (columns) shown in the following table\cite{C3}. 

\begin{center}
\begin{tabular}{c|c}
\hline
Variable & Definition\\
\hline
survival & Survival\\
pclass & Ticket class\\
PassengerId & ID of each passenger \\
sex & gender   \\
Age & Age in years   \\
ticket & Ticket number  \\
sibsp & number of siblings / spouses aboard the Titanic  \\
\hline
\end{tabular}
\end{center}

\begin{center}
\begin{tabular}{c|c}
\hline
Variable & Definition\\
\hline
parch & number of parents / children aboard the Titanic   \\
name & name of each passenger \\
fare &  Passenger fare	  \\
cabin &  Cabin number   \\
embarked &  Port of Embarkation \\
\hline
\end{tabular}
\end{center}

\par 
Survival attribute for this example will be the label for classification. It has two values 0 for not survived and 1 for survived. Pclass attribute has 3 possible values 1 for upper class, 2 for middle class, and 3 for lower class. Age attribute is fractional if less than 1. If the age of a passenger is estimated, is it in the form of ``xx.5''. Sibling defined in this data set is brother, sister, stepbrother, and stepsister. Spouse defined in this data set is husband and wife. Parent defined in this data set contains mother and father. Child in this data set includes daughter, son, stepdaughter, and stepson\cite{C3}. 

\par 
After defining the data value and storing the data in a algorithm-readable format, historians should process the missing values and noise which is the most significant step before analyzing or mining the data by an algorithm. Noise usually refers to non-systematic error. Such error is not caused by the algorithm or the classifier system. It is from the training dataset. For example, two tuples has the identical values in all attributes, but their label is different. It causes the inadequate attributes. To deal with the noise, the historians could delete such tuples. 

\par 
If there is missing value in an attribute, the mean value is usually used as the substitution. However there is a more technological approach to fill in an unknown value by using the information provided by context. For example, the historians could use a Bayesian formalism to figure out the probability of a possible value say $A_i$ in attribute A. In addition, decision tree approach is another way to determine the missing value. Assume $C_s$ is a subset of C consisting of an attribute and the label. the historians could construct a decision tree based on this subset and predict the missing value using the tree model.

\section{Analysis and implementation of ID3 algorithm}
The label (survival attribute) of the dataset is binary value. All the attributes contain discretely numerical value. Therefore, ID3 algorithm is the best algorithm to be employed for mining and analyzing the dataset. 

\par 
ID3 algorithm is well known as Iterative Dichotomiser 3 algorithm invented by Ross Quinlan. It is used to generate a decision tree recursively from a dataset. Then, the output decision tree could be used as a model to predict (classify) any input instance as which class or group it belongs to. The ID3 algorithm starts at the original dataset as the root node. Then in the iteration (recursion), the algorithm calculates the entropy of the label and the information gain of each unused attribute. The algorithm selects the attribute which has the greatest information gain and separates the dataset into multiple subset according to possible value of this chosen attribute. Each subset of the original dataset is an inner node in the final decision tree. ID3 algorithm continues to call itself (recursion) until two conditions are satisfied. First condition is that every element in the subset belongs to the same class, the subset denoted as the leaf of the decision tree is marked as the name of that class. Second condition is that there are no more attributes to be selected, but the examples still do not belong to the same class, then the node is turned into a leaf and labelled with the most common class of the examples in the subset\cite{C4}. 

\par
The concept entropy this paper discussed above is derived from 
Physics, which is a measure of how much chaos or uncertainty of the system. In big data, it refers to the amount of uncertainty of the given dataset. Entropy is usually represented by H(S). It could be calculated from the probability of each element in the label of a dataset. 
$$H(S) = \sum_{ x \in X } -p(x)log_2p(x)$$
Where S is the current dataset for which entropy is being calculated, capitalized X is the set of classes in the label of the dataset, and $p(x)$ is the probability of each element in class $x$. 
Information gain is a concept to measure how much more information we could acquire from the dataset if we analyze the data one further step. The greater information gain is, the larger uncertainty or the more information we could obtain from the dataset if mining the data more. Information gain is usually denoted by $IG(A;S)$, where A is an attribute and S is the current dataset. It could be calculated from the following formula.
$$IG(A;S)= H(S) - H(A|S)$$
$$IG(A;S)= H(S) - \sum_{a \in A}\frac{S_a}{S} H(S_a)$$
where $H(S)$ is the entropy of dataset S, a is an element in attribute A, $S_a$ is a collection of tuples whose A attribute value is a. 

\section{Conclusion}
This paper introduces the Titanic story and how historians preprocess the data. The ID3 algorithm is also introduced for historians to employ on their research. As a historian, big data is an opportunity but also a big obstacle for the future researches. But at least, it is a good start in big data for all historians.

\bibliographystyle{ACM-Reference-Format}
\bibliography{report} 

\end{document}


